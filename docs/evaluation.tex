\subsection{Evaluation}
\label{evaluation}
The project's development process and end-product will be discussed in this section.
Tsukiji will be analysed in terms of code quality and how much requirements have been fulfilled.

%3 pages
%Talk about planning, development style
\subsubsection{Development process}
The planning of the development was to spend 2 weeks researching the given problem and then spend around 6 weeks developing a solution and document the decisions made during those weeks on the go.
This structure was uphold throughout the lifespan of the project.
It was notable that documentation began lacking halfway, but this was quickly corrected to make sure every decision was written down as accurate as possible.\\
\\
Due to agile development methodology, it was difficult to plan every sprint beforehand, since requirements could change and the focus of the project could diverge from what was decided in the first meeting with the client.
Regardless of this, the Scrum way of developing proved to be very useful.
Exactly because of the volatile nature of the project, creating additions to the software on a week to week basis turned out perfect. 
Along the way decision such as ignoring people possibly cheating the system, the protocol design could be simplified without having lost time on preemptively implementing a structure for such a protocol.\\
\\
%Communication
A large help to the development of Tsukiji was that the team always worked together. 
Since it consisted out of only 2 developers, it became very easy to discuss issues and decide to pick a certain library or protocol.
The developers sat next to each other which eliminated that delay of response that email brings. 
The load of work was divided such that both members were responsible for a certain feature of the software.
Whenever a member finished his implementation for the current sprint, he immediately documented what he had written, while the other member finished his code.

%Talk about fullfillment of requirements
\subsubsection{Fulfillment of requirements}
\label{requirementsfill}
In chapter \ref{requirements}, the requirements of this project are discussed. 
One of the most important parts of a piece of software is whether it complies with the requirements given by the client.
This section will follow along with the MoSCoW list given in chapter \ref{requirements}, beginnen with the 'Must have'-requirements and ending with the 'Would have's. \\
\\
Every 'Must have'-requirement has been fulfilled by the implementation.
Users are able to place offers and respond to offers. 
The response is currently handled by a trading engine, that matches the asks with bids, given the same quantity and value.
Apart of a peerlist that is distributed with the software, the system is completely decetralised. 
Every peer can find another peer through the peerlist.
Passing messages happens with the gossip protocol, which does not require a message-server.
Authentication is done through public and private keys.
This also does not require a central point.

The scalability is tested with a simulation on a single computer.
This test showed that messages can easily be handled in the range of 1000s of users and that duplicate messages are not passed along when received.

All of these tests were preformed in a linux terminal, so the requirement of accepting command line input has been fulfilled as well.\\
\\
The only 'Should have'-requirement has been fulfilled.
Users are able to spend real money (currently any currency supported by PayPal) on the good traded in tsukiji.
We have chosen not to use BitCoin for this since real money is not far more stable and is far less likely to suddenly disappear. 
The currency used by the software should not impact a person's ability to make trades, so if BitCoin ever stops existing, Tsukiji will not fall as well.\\
\\
The current implementation fulfills no 'Could have'- or 'Would have'-requirements.
Implementing a User Interface takes a lot of time while not increasing the functionality.
Since Tsukiji is a proof of concept, we wanted to create as much functionality as possible in the time given.
This lead to the decision to focus on other features than a GUI, and kept interaction with the program in a terminal.

The 'Would have'-requirements are mainly security issues.
These problems can certainly be solved, but would require a lot of time to achieve something that is outside the scope of this project.
Tsukiji was created to show that it is possible to create a marketplace without a central authority, it was not created to protect users of an online marketplace from fraud.
This could possibly be addressed in a different project, using the existing code-base.

The privacy requirement is party fulfilled. Users do not directly have an identifier bound to their real identity.
It is simply an MD5 hash sent across the network.
The offers created can be tracked down to the hash, and to the IP that sent the message.
A user spoof his IP whenever he is creating trades to stay hidden, but this is its own responsibility.
Besides that, a possibly privacy breach could be PayPal.
To avoid using ones real account, it is possible to create a new PayPal account on a fake email address and use that to make transactions.
The bank accounts linked to PayPal are not visible in transactions, so the trades will not give away a users identity.\\
\\
The past paragraphs show that all the crucial requirements have been met.
This means that the software created as sufficient for what the client asked for.

\subsubsection{Known issues}
The current implementation of Tsukiji has a couple of issues that have not been addressed.
Section \ref{requirementsfill} already touched on a couple of missing features such as perfect anonymity and security.
There is another major factor that is problematic for the current state of the protocol.

Currently, a user can lie about his transactions.
It is possible to pretend to have a certain amount of points for sale, while not actually having any at all.
This is because Alice has no way to check the transactions Bob has made.
Alice would only be able to see that a certain bid or ask is no longer in the list of offers.
If Bob spoofed a message that says that a certain transaction has been made, Alice has no way to verify the truth of that.
This creates a trust-issue that is currently not resolved.
Section \ref{recommendations} provides a possible solution for this problem.\\
\\
Another known issue of Tsukiji is that it is risky to make a large transaction, since incremental payments (see section 7.5) have not been implemented.
Whenever a user now send someone else money to buy points, there is no way to force the other party to actually send the points.
This creates a situation where only small amounts of points will be traded at once and will quickly clutter the marketplace with many small offers.\\
\\
These issues can be resolved, but not within the time given for this project.
It would be interesting to extend Tsukiji to support the solution for these issues, since that would make it possible for this proof of concept to be used a real piece of software.


%trust issues
%security
%anonimization

%Talk about test coverage(picture!) and SIG feedback
\subsubsection{Code quality}
