\section{Recommendations}
\label{recommendations}
The goal of this project is to create a proof of concept in a limited amount of time.
Because of this, there are multiple possible features that have been put aside over in favour of other features deemed more essential.
In this chapter we suggest some recommendations for further progress on the project.
They can be added to this proof of concept to build Tsukiji into a fully fledged user program, or to further the research of possibilities of a decentral market.

\subsubsection*{$\bullet$ Finishing Incremental Payment}
Section \ref{incpay} covers the concept of incremental payments. One of the issues discussed was identifying which payment on Paypal corresponds with which trade on Tsukiji.
One possible solution would be sending messages over the network of Tsukiji to the seller that a payment has been made.
This message would contain a chain-id and an index of the transaction of in that chain.
These id's would then have to be added a notification in the PayPal transaction and be extracted when the seller reads its transactionlist.

\subsubsection*{$\bullet$ Graphical User Interface}
A big improvement to Tsukiji would be to add a graphical user interface. 
This would make it much easier to use the program and understand what is going on.
Graphs and lists with variable ordering can vastly improve the format in which information is displayed.
Additionally, with a clear interface, Tsukiji would be far more inviting to the average user than a command-line tool.
Command line input brings multiple challenges, such as requiring the user to know the right commands and the syntax of those instructions.
A GUI can represent those instructions with a more intuitive interface.

\subsubsection*{$\bullet$ Anti-spoofing}
Section \ref{knownissues} covered the issue of people referencing false transactions.
They are pretending to have more tradable goods that they in reality have.
A possible solution is to implement a block chain like BitCoin has done.
Their structure requires the whole userbase to agree on a set of transactions made.
These trades are hashed and used to determine whether the following set of transactions are real.
For a more in-depth explanation of the block-chain verification, we refer to the BitCoin paper \cite{bitcoin}.

\subsubsection*{$\bullet$ Trading large quantities}
The current implementation of Tsukiji only supports trading a single unit of an item.
It is very possible that users want to trade a larger quantity at a time.
This will also reduce the amount of offers needed to perform a certain trade, decreasing clutter in the list of open offers.