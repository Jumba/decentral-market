\section{Problem definition}
\epigraph{\textbf{mar - ket - place}

1. An open square or place in a town where markets or public sales are held.

2. The world of trade or economic activity.}{Merriam-Webster}

To combat the issues associated with centralised marketplaces, Tsukiji aims to be a fully decentralised marketplace.

A marketplace is a place where public sales are held.
For our purposes, a marketplace is a place where people can place bids or asks.
A bid is placed when a buyer wants to buy goods for a set price.
An ask is placed when a seller wants to sell goods for a set price.

Using the example of BitTorrent communities, a transaction would occur as follows: a seller places an "ask", wanting to sell 500 MBit for €3.
An interested buyer sees the ask and requests a trade to be made.
The buyer and seller exchange information necessary for the transaction to occur (e.g. bank account numbers).
A trade is made and the original ask is taken down.
A similar transaction occurs when a "bid" is placed.

Current marketplaces are centralised.
Usually, this type of centralised system follows the client-server model, where one server serves a large amount of clients.
In this model, information passes through one central point.
This means that if this central point fails, the entire marketplace is shut down.
A decentralised system operates without such a single point.
Instead of a central server connected to several users, peers are connected to each other.
This follows the peer-to-peer model.
If a single peer fails, this might impact the marketplace, but it will not shut it down.

For a network of such peers to exist, peers first have to find each other.
Peer discovery is generally done by exchanging information about other peers.
When a peer first signs on, it knows of no other peers.
We can bootstrap its network by having them connect to a predefined set of super peers, who are known to be well connected.
From there, they can exchange information about other peers.