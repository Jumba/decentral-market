\subsection{Sprint 3: Testing at scale}
The theme of the third sprint is testing.
Testing is an important part of software development.
It gives the developers certainty over the quality of their program.
We felt like this sprint would be the perfect time to start testing.
It is better to start testing as early as possible.
We started relatively late compared to, for example, the Test Driven Development methodology, where tests are written even before the application logic is written.
However, it has allowed us to quickly iterate on Tsukiji in the early sprints.
At this stage, Tsukiji had reached a level of maturity where logic starts to become significantly more complex.
Tests give us assurance our program works correctly for the test cases we have defined.
There are two forms of testing focused on during this sprint: unit testing and scalibility testing.
Another form of testing that isn't focused on during this sprint is integration testing.
The purpose of integration testing is combine the components of a system and to test whether or not they produce the desired result.
However, at this point tsukiji is still sufficiently small that we believe producing such tests would yield little benefit.

\subsubsection{Unit testing}
Unit testing is the testing of small units of code, ideally the smallest block of code possible.
For our purposes, the smallest block of code is an individual function or class method.
Each unit should be tested in isolation, i.e. each test case is independent of each other.

\subsubsection{Scalibility testing}
The other form of testing focused on during this sprint is large scale testing.
The purpose of our large scale test is to test how functional the network remains with a large amount of peers.