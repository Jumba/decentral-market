\section{Introduction}
\epigraph{The Internet is becoming the town square for the global village of tomorrow.}{Bill Gates}

The appearance of online marketplaces in the late 90s marked the beginning of a new era for consumers.
The connection between supplier and consumer is closer than ever.
Even trading between individuals has become commonplace.
Sites like ebay, Amazon and Alibaba all provide consumer-to-consumer, business-to-consumer and even business-to-business services.
Compared to traditional shopping, these platforms are easier to use, cost less time, have a wider range of products available and are better at matching buyer and seller.

Another example of such a marketplace is currency exchange platforms.
In recent years, Bitcoin has become of great interest to traders.
With the rise of interest in Bitcoin, the rise of Bitcoin exchange markets comes naturally.
The most famous of Bitcoin exchanges is the now defunct Mt.Gox.
Mt.Gox was plagued by problems early on that eventually led to its bankruptcy.
With its downfall, it took a large amount of funds with it.
The trust users have put into the institution was violated.

% Third example: stock market?

Similar stories exist all over the place. Hacking attempts, government shutdowns, or simply incompetent companies.
These are all common among marketplaces of all kinds.
In this report we present \textit{tsukiji}, a decentralized marketplace.
This proof-of-concept marketplace has no central server bottleneck, no central point of trust, full self-organisation, and unbounded scalability.

%Torrents
Motivation and inspiration for creating Tsukiji originally came from Bittorrent communities.
The sharing of files has become more and more popular over the years.
Napster, Kazaa and LimeWire no longer exist, but one protocol has survived: BitTorrent.
BitTorrent is a peer-to-peer file sharing protocol.
Instead of downloading from a central server, each peer downloads from and uploads to other peers.
It offered a solution to the bottleneck of existing websites.

With more and more people using BitTorrent, the amount of people downloading a certain file increases.
Ideally, every peer should upload at least as much as they download.
Sadly, this does not happen in reality.
The amount of users only downloading and not uploading is significantly higher than people uploading at least as much as they download.
There is no incentive for an individual to upload, but with few peers who upload the download speed is significantly hampered.
To solve this issue, uploaders decided to work together in separate, private groups that only allow people with a good upload/download ratio.

This seems like a good solution to motivate peers to actively upload, but a couple of issues occur.
With peers motivated to upload, a certain oversupply of uploading rises.
A peer with a low speed connection, has a hard time competing with many high-speed connections within the group.
This peer now loses reputation because of competition, even though they are putting in the effort to upload.

Unpopular files form another issue.
If someone downloads a large file and wants to upload it to increase his reputation, it is possible that there are no users interested in downloading that file.
This gives the peer a large deficit in his reputation that he cannot solve by uploading his unpopular taste.

Lastly, torrenting may induce with legal issues. In some countries, it is illegal to upload files, whereas downloading is allowed.
Citizens of such countries would like the high speed download of private communities, but cannot participate since they are legally not allowed to share whatever they have downloaded.

To solve these issues, peers need a way te obtain reputation in a way other than uploading.
We propose that users can trade reputation points for currency.
Reputation is still gained by uploading.
Peers can create offers on a marketplace to buy or sell reputation points.
This will give peers the incentive to upload data to the swarm.
At the same time, it will still be possible for peers to have access to the community by spending money and buying into a community.
This will lead to a more accessible and scalable community than traditional private torrent communities.

Keeping in line with BitTorrent's philosophy, Tsukiji will be fully decentralised and peer-to-peer.
With secure cryptography, the users will be able to keep their real identity hidden, while still being able to make trades with their network-alias.
Certain users will attempt to upload as much as possible to sell large amounts of reputation, keeping the network attractive with very high download speeds.
With no central authority, the market is completely free to decide on a price for their reputation.
There is no added value of intermediaries since peers can directly sell to each other, keeping the price as low as possible.

% Insert description of each chapter, e.g. "In chapter 5 we will talk about wooly elephants."