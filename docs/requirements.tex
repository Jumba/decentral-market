\subsection{Requirements}
\label{requirements}

The goal of Tsukiji is to create a decentralised market where users can trade commodities.
Tsukiji is primarily a proof-of-concept, i.e. it should showcase the technologies and viability of a decentralised market.
While Tsukiji is a proof-of-concept, this does not mean we have to forego a usable application altogether.
The requirements are summarized using the MoSCoW method.
This method assigns a priority to the requirements.
The importance of a requirement influences the planning during the project.

\subsubsection{Must Have(s)}
\begin{itemize}
\item Place an offer

The basics of a marketplace should work.
This means users should be able to place an offer, i.e. an ask or bid.

\item Respond to an offer, facilitating a trade

They should also be able to see offers from other users.
From an offer, a trade can be initiated.

\item Decentralised

The system should be decentralised.
The lack of a clear single point of failure is the theme of this project.
Tsukiji can take inspiration from peer-to-peer networks such as BitTorrent.

\item Peer discovery

Paired with a decentralised system comes some form of peer discovery.

\item Scalable to thousands of users

It should scale to thousands of users.
If our goal is to create a usable application, the network should be able to handle many connections.
The choice for thousands of users (as opposed to hundreds or millions) is motivated by the scale of other applications, such as Tribler \cite{tribler}. %i.e. Tribler

\item Command line input

A good user interface is not a priority of this project.
While it is important for a usable application, a lot of time would have to be devoted to this.
We believe this time is better spent on other areas of the application.
The user interface will feature the bare minimum, i.e. a command line interface.
\end{itemize}

\subsubsection{Should Have(s)}
\begin{itemize}
\item Trade using real money

A user should be able to trade real commodities using real money.
However, it is not a must, because virtual currencies and commodities are good enough to show that the proof of concept works.
\end{itemize}

\subsubsection{Could Have(s)}
\begin{itemize}
\item Nice user interface

As explained above, a nice user interface is not a must.
However, a better user experience is still an advantage as it allows the general public to use the application also.
\end{itemize}

\subsubsection{Would Have(s)}
\begin{itemize}
\item Privacy

It would be nice to ensure the privacy of the users.
In order to achieve this, strong encryption of messages is required.
Further thought is also necessary on what information is exposed.
At the very least, no personal information should be stored.
Networking information, such as ip addresses, are exposed.

\item Protection against hostile attacks

It would be nice to have a network resistant against hostile takeover.
Having a decentralised system plays a big role in this.
However, other attack vectors could also be possible.
The main weapon against this is a well thought out protocol.
However, to keep this project viable within the timespan given, we will assume there to be no active attackers.

\item Anti-spoofing

Spoofing of message would of course form a big problem in a trading network.
However, since this is a solved problem, the focus of the project should be on other areas.
For now, it is assumed that every message is legitimate.
\end{itemize}